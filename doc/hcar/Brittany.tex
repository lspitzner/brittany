% Brittany-LB.tex
\begin{hcarentry}[updated]{Brittany}
\report{Lennart Spitzner}%11/17
\status{work in progress}
\makeheader

Brittany is a Haskell source code formatting tool. It is based on
ghc-exactprint and thus uses the ghc parser, in contrast to tools based on
haskell-src-exts such as hindent or haskell-formatter.

The goals of the project are to:

\begin{compactitem}
\item support the full ghc-haskell syntax including syntactic extensions;
\item retain newlines and comments unmodified (to the degree possible when
  code around them gets reformatted);
\item be clever about using horizontal space while not overflowing it if it
  cannot be avoided;
\item have linear complexity in the size of the input text / the number of
  syntactic nodes in the input.
\item support horizontal alignments (e.g. different equations/pattern matches
  in the some function's definition).
\end{compactitem}

In contrast to other formatters brittany internally works in two steps:
Firstly transforming the syntax tree into a document tree representation,
similar to the document representation in general-purpose pretty-printers such
as the \emph{pretty} package, but much more specialized for the specific
purpose of handling a Haskell source code document. Secondly this document
representation is transformed into the output text document. This approach
allows to handle many different syntactic constructs in a uniform way, making
it possible to attain the above goals with a manageable amount of work.

Brittany is work in progress; currently only type signatures and function
bindings are transformed, and not all syntactic constructs are supported.
Nonetheless Brittany is safe to try/use as there are checks in place to ensure
that the output is syntactically valid.

Brittany requires ghc-8.*, and is available on Hackage and on Stackage.

\FurtherReading
\begin{compactitem}
  \item \url{https://github.com/lspitzner/brittany}
\end{compactitem}
\end{hcarentry}
